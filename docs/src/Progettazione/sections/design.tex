\section{I colori}

Come stabilito in fase di pianificazione, l'applicazione verterà su un colore rosso-arancio, in quanto solitamente associato alla nota di Do. 

Si è, quindi, generata una \emph{palette} di colori partendo da un colore arancio scuro, simile al colore dei cachi. Si sono scelti quattro colori seguendo la ``regola'' della tetrade cromatica, selezionando dei colori con una distanza di trenta gradi circa (sulla ruota cromatica) dal colore principale.

Si veda la tabella \ref{tab:palette} per avere dei riferimenti visivi sui colori scelti. Per ogni colore, sono presentate quattro tinte diverse (escludendo il colore ``puro'', mostrato in posizione centrale) e sono riportati i vari codici in esadecimale. Inoltre, è possibile osservare la resa sia di un testo bianco che di uno nero sulle varie tinte.

\renewcommand{\arraystretch}{1.5}
\begin{table}[H]
	\centering
	\caption{\emph{Palette} dei colori su cui è basato il \emph{design} di \ProjectTitle{}.}
	\label{tab:palette}
	\begin{tabular}{lccccc}
		\hline
		{\footnotesize Colore primario:}       & \cellcolor[HTML]{FF9E6B}{\begin{tabular}[c]{@{}c@{}}{\color[HTML]{FFFFFF} \texttt{\#FF9E6B}}\\ \texttt{\#FF9E6B}\end{tabular}} & \cellcolor[HTML]{FF8C4F}{\begin{tabular}[c]{@{}c@{}}{\color[HTML]{FFFFFF} \texttt{\#FF8C4F}}\\ \texttt{\#FF8C4F}\end{tabular}} & \cellcolor[HTML]{E55100}{\begin{tabular}[c]{@{}c@{}}{\color[HTML]{FFFFFF} \texttt{\#E55100}}\\ \texttt{\#E55100}\end{tabular}} & \cellcolor[HTML]{802D00}{\begin{tabular}[c]{@{}c@{}}{\color[HTML]{FFFFFF} \texttt{\#802D00}}\\ \texttt{\#802D00}\end{tabular}} & \cellcolor[HTML]{571E00}{\begin{tabular}[c]{@{}c@{}}{\color[HTML]{FFFFFF} \texttt{\#571E00}}\\ \texttt{\#571E00}\end{tabular}} \\ \hline
		{\footnotesize Colore secondario (1):} & \cellcolor[HTML]{FFC56B}{\begin{tabular}[c]{@{}c@{}}{\color[HTML]{FFFFFF} \texttt{\#FFC56B}}\\ \texttt{\#FFC56B}\end{tabular}} & \cellcolor[HTML]{FFB94F}{\begin{tabular}[c]{@{}c@{}}{\color[HTML]{FFFFFF} \texttt{\#FFB94F}}\\ \texttt{\#FFB94F}\end{tabular}} & \cellcolor[HTML]{E58B00}{\begin{tabular}[c]{@{}c@{}}{\color[HTML]{FFFFFF} \texttt{\#E58B00}}\\ \texttt{\#E58B00}\end{tabular}} & \cellcolor[HTML]{804D00}{\begin{tabular}[c]{@{}c@{}}{\color[HTML]{FFFFFF} \texttt{\#804D00}}\\ \texttt{\#804D00}\end{tabular}} & \cellcolor[HTML]{573500}{\begin{tabular}[c]{@{}c@{}}{\color[HTML]{FFFFFF} \texttt{\#573500}}\\ \texttt{\#573500}\end{tabular}} \\ \hline
		{\footnotesize Colore complementare:}    & \cellcolor[HTML]{6FABEF}{\begin{tabular}[c]{@{}c@{}}{\color[HTML]{FFFFFF} \texttt{\#6FABEF}}\\ \texttt{\#6FABEF}\end{tabular}} & \cellcolor[HTML]{4D8DD5}{\begin{tabular}[c]{@{}c@{}}{\color[HTML]{FFFFFF} \texttt{\#4D8DD5}}\\ \texttt{\#4D8DD5}\end{tabular}} & \cellcolor[HTML]{0C4D95}{\begin{tabular}[c]{@{}c@{}}{\color[HTML]{FFFFFF} \texttt{\#0C4D95}}\\ \texttt{\#0C4D95}\end{tabular}} & \cellcolor[HTML]{012853}{\begin{tabular}[c]{@{}c@{}}{\color[HTML]{FFFFFF} \texttt{\#012853}}\\ \texttt{\#012853}\end{tabular}} & \cellcolor[HTML]{001B39}{\begin{tabular}[c]{@{}c@{}}{\color[HTML]{FFFFFF} \texttt{\#001B39}}\\ \texttt{\#001B39}\end{tabular}} \\ \hline
		{\footnotesize Colore secondario (2):} & \cellcolor[HTML]{64EFC5}{\begin{tabular}[c]{@{}c@{}}{\color[HTML]{FFFFFF} \texttt{\#64EFC5}}\\ \texttt{\#64EFC5}\end{tabular}} & \cellcolor[HTML]{42D6A9}{\begin{tabular}[c]{@{}c@{}}{\color[HTML]{FFFFFF} \texttt{\#42D6A9}}\\ \texttt{\#42D6A9}\end{tabular}} & \cellcolor[HTML]{00976A}{\begin{tabular}[c]{@{}c@{}}{\color[HTML]{FFFFFF} \texttt{\#00976A}}\\ \texttt{\#00976A}\end{tabular}} & \cellcolor[HTML]{00543B}{\begin{tabular}[c]{@{}c@{}}{\color[HTML]{FFFFFF} \texttt{\#00543B}}\\ \texttt{\#00543B}\end{tabular}} & \cellcolor[HTML]{003928}{\begin{tabular}[c]{@{}c@{}}{\color[HTML]{FFFFFF} \texttt{\#003928}}\\ \texttt{\#003928}\end{tabular}} \\ \hline
	\end{tabular}%
\end{table}
\renewcommand{\arraystretch}{2}

Ai precedenti colori, vanno poi aggiunti i colori bianco (\colorbox[HTML]{FFFFFF}{\texttt{\#FFFFFF}}), nero (\colorbox{black}{\color{white}\texttt{\#000000}}) e grigio all'80\% (\colorbox[HTML]{333333}{\color{white}\texttt{\#333333}}), utilizzati per contrastare i colori più accesi e per il testo dell'applicazione.