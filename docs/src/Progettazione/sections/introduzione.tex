L'\textit{hypermedia} \ProjectTitle{} ha l'obiettivo di avvicinare gli utenti al mondo della musica e di aiutarli a migliorare le loro conoscenze riguardanti questa arte.
\section{I concetti}
L'applicazione conterrà i seguenti concetti:
\begin{itemize}
	\item Concetti teorici
	\item Strumenti
	\item Accordi
	\item Quiz
\end{itemize}

L'applicazione è quindi divisa in sezioni (tante quante i singoli concetti). Tali sezioni sono ulteriormente divise in ``unità'' all'interno delle quali ogni argomento sarà presentato attraverso una combinazione di testo, immagini, video e audio.

L'applicazione avrà una pagina iniziale che ha lo scopo di introdurre il concetto di ``musica'', nonché di presentare all'utente delle istruzioni basilari sull'uso del sistema.

\ProjectTitle{} sarà introdotto da una semplice \textit{splash screen} che permetterà all'utente di accedere più gradualmente all'applicazione. Si accompagnerà il tutto con un manuale utente.

Il sistema conterrà una sezione per ogni unità che possa permettere all'utente di mettersi alla prova tramite domande a risposta multipla o ``minigiochi''.
\subsection{Definizione dei concetti}
\paragraph{Concetti teorici} I concetti teorici sono alla base dell'interfaccia con il mondo della musica. Sono suddivisi in due unità (più propriamente dette ``livelli''): \textit{teoria di livello base} e \textit{teoria di livello intermedio/avanzato} in cui, rispettivamente, saranno presentati i concetti necessari alla comprensione dei contenuti e saranno presentati dei concetti di approfondimento. Gli argomenti saranno trattati in modo da essere facilmente comprensibili dagli utenti che non hanno conoscenze in questo campo.
\paragraph{Strumenti} Come tutte le arti, la musica necessita di una grande abilità pratica. Tramite le varie unità (che si identificano con gli strumenti presenti nell'applicazione, ovverosia: batteria, basso, chitarra e pianoforte) in cui è diviso questo concetto l'utente può interfacciarsi sulle tecniche base, intermedie e avanzate dello strumento scelto.
\paragraph{Accordi} Una parte degli strumenti prevede la possibilità di suonare degli accordi. Nelle varie unità (che, anche in questo caso, si identificano con gli strumenti) di questo concetto, l'utente potrà imparare a suonare degli accordi con il proprio strumento.
\paragraph{Quiz} Per ogni unità dei concetti precedentemente definiti, sarà presente un'unità di \textit{test} che consentirà all'utente di verificare le proprie conoscenze. Non è necessario superare un test per procedere nell'utilizzo dell'applicazione, ma è consigliato.
\section{I task}
\begin{enumerate}
	\item Conoscere la teoria che c'è dietro l'arte della musica
	\item Conoscere le tecniche basilari di almeno uno strumento
\end{enumerate}
\begin{figure}[H]
	\label{fig:tasks}
	\caption[I task]{I task di \ProjectTitle{}. Sono stati disposti orizzontalmente (da sinistra verso destra) affinché potessero facilmente essere inseriti in questo documento.}
	\resizebox{\textwidth}{!}{%
		\begin{tikzpicture}[%
		sibling distance=10em,
		level distance=8em,grow=right,
		every node/.style = {shape=rectangle, align=center, inner sep=1em,draw},
		edge from parent/.style={draw,thick,-latex},
		root/.style={align=center, thick, rounded corners, font=\bfseries\scshape\color{white}, fill=mainColorDark},
		level 1/.style={level distance=8em,sibling distance= 20em, fill=mainColorDark!60},
		level 2/.style={level distance=18em,sibling distance= 10em, fill=mainColorDark!30},
		final/.style = {thin, fill=mainColorDark!60, ellipse, fill=mainColorDark!5}
		]
			\node[root] (root) {Entrare nel mondo\\della musica}
			child {
				node[level 1] (c1) {Conoscere i\\concetti teorici} 
				child {
					node[level 2] (c11) {Conoscere i\\concetti di base}
					child {
						node[final] (c111) {Trasferire conoscenze\\mediante testo,\\immagini e audio}
					}
				}
				child {
					node[level 2] (c12) {Conoscere i\\concetti avanzati}
					child {
						node[final] (c121) {Trasferire conoscenze\\mediante testo\\e immagini}
					}
				}
			}
			child { 
				node[level 1] (c2) {Conoscere uno\\strumento}
				child {
					node[level 2] (c21) {Conoscere le\\tecniche avanzate\\(accordi)}
					child { 
						node[final] (c211) {Trasferire conoscenze\\sugli accordi mediante\\immagini, video e audio} }
				}
				child { 
					node[final] (c22) {Trasferire conoscenze\\sullo strumento scelto\\mediante testo e immagini} 
				}
			};
		\end{tikzpicture}%
	}
\end{figure}