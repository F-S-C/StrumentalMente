\section{I linguaggi}
Il sistema \ProjectTitle{} è scritto utilizzando l'\textbf{HTML5} (il che porta
all'utilizzo del \textbf{CSS3} per definire le varie regole di stile) e il
\textbf{JavaScript}. L'intera applicazione è basata su \emph{Node.js},
principalmente per il pacchetto \textbf{Electron}, che consente di utilizzare i
precedenti linguaggi, in combinazione con il motore di rendering
\emph{Chromium}, per creare delle applicazioni \textit{desktop}.

\section{Il codice}
Il codice sorgente di \ProjectTitle{} è fondamentalmente suddiviso in tre file
principali (escludendo i file HTML e CSS utilizzati per creare l'aspetto grafico
dell'aplicazione). Tali file sono:
\begin{description}
	\item[\texttt{app.js}] È il file principale dell'applicazione. Contiene il
	processo principale (``\emph{main process}'') di Electron e gestisce,
	quindi, tutti gli eventi principali di \ProjectTitle{}, tra cui:
	\begin{itemize}
		\item L'apertura dell'applicazione e il \emph{rendering} della prima
		finestra
		\item La chiusura dell'applicazione e le relative peculiarità di alcuni
		sistemi operativi (si pensi alla possibilità di ricreare la finestra
		appena chiusa su MacOS)
		\item L'apertura di finestre di dialogo
		\item L'apertura di finestra secondarie
	\end{itemize} 
	\item[\texttt{render.js}] È il file principale del \emph{rendering}
	dell'applicazione. Contiene tutte le funzioni da eseguire nel
	``\emph{rendering process}'' (processo di \emph{rendering}) di Electron.
	\item[\texttt{main.js}] Contiene tutte le funzioni che servono a dare
	dinamicità alle pagine HTML dell'applicazione.
	\item[\texttt{quiz.js}]
	\item[\texttt{accordo.js}]
\end{description}