Affinchè si possa avviare \ProjectTitle{} dal codice sorgente è necessario che siano installati \texttt{npm} e \texttt{Node.js}. Successivamente, è necessario installare le dipendenze del progetto, eseguendo i seguenti comandi nella \textit{directory} in cui è situato il codice sorgente (cartella \texttt{src/} della \textit{repository}):

\begin{lstlisting}
npm install --save-dev electron
npm install --save-dev mousetrap
npm install --save-dev node-localstorage
\end{lstlisting}

È possibile installare \textit{Electron Packager} per poter compilare un eseguibile (non è necessario per avviare l'applicazione) utilizzando il comando:

\begin{lstlisting}
npm install -g electron-packager
\end{lstlisting}

Dopo aver installato tutte le dipendenze, è possibile avviare l'applicazione utilizzando il comando seguente all'interno della \textit{directory} del codice sorgente:

\begin{lstlisting}
npm start
\end{lstlisting}

\section{Il Makefile}

Per comodità, si è creato un \texttt{Makefile} per eseguire facilmente tutte le operazioni sul codice. Il comando \lstinline|make| è da eseguire all'interno della \textit{directory} principale della \textit{repository}.

Il file \texttt{Makefile} contiene le seguenti regole:
\begin{description}
	\item[\texttt{\textit{(all)}}] Avvia \ProjectTitle{}.
	\item[\texttt{install}] Installa le dipendenze del progetto ed \textit{Electron Packager}.
	\item[\texttt{start}] Avvia \ProjectTitle{}.
	\item[\texttt{deploy}] Compila \ProjectTitle{} in un eseguibile per varie piattaforme. Al momento le piattaforme per cui la generazione è inclusa in questa regola sono:
	\begin{itemize}
		\item Windows (x64)
	\end{itemize}
	\item[\texttt{docs}] Ricompila tutta la documentazione partendo dai sorgenti \LaTeX.
	\item[\texttt{jsdoc}] Ricompila la documentazione del codice in un file \LaTeX.
	\item[\texttt{bib2html}] Converte la bibliografia in formato \textsc{Bib}\TeX{} in HTML e la importa automaticamente nell'apposita sezione della pagina "about".
\end{description}