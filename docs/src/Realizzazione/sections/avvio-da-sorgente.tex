Affinchè si possa avviare \ProjectTitle{} dal codice sorgente è necessario che
siano installati \texttt{Node.js} e \texttt{yarn}. Successivamente, è necessario
installare le dipendenze del progetto, eseguendo il seguente comando nella
\textit{directory} in cui è situato il codice sorgente (cartella \texttt{src/}
della \textit{repository}):

\begin{lstlisting}
yarn install
\end{lstlisting}

Dopo aver installato tutte le dipendenze, è possibile avviare l'applicazione
utilizzando il comando seguente all'interno della \textit{directory} del codice
sorgente:

\begin{lstlisting}
yarn start
\end{lstlisting}

\section{Il Makefile}

Per comodità, si è creato un \texttt{Makefile} per eseguire facilmente tutte le
operazioni sul codice. Il comando \lstinline|make| è da eseguire all'interno
della \textit{directory} principale della \textit{repository}.

Il file \texttt{Makefile} contiene le seguenti regole:
\begin{description}
	\item[\texttt{\textit{(all)}}] Avvia \ProjectTitle{}.
	\item[\texttt{install}] Installa le dipendenze del progetto (con l'opzione
	\texttt{--force} che forza l'eventuale reinstallazione).
	\item[\texttt{start}] Avvia \ProjectTitle{}.
	\item[\texttt{deploy}] Compila \ProjectTitle{} in un eseguibile per la
	piattaforma su cui il comando è stato eseguito.
	\item[\texttt{docs}] Ricompila tutta la documentazione partendo dai sorgenti
	\LaTeX.
	\item[\texttt{jsdoc}] Ricompila la documentazione del codice in un file
	\LaTeX.
	\item[\texttt{bib2html}] Converte la bibliografia in formato
	\textsc{Bib}\TeX{} in HTML e la importa automaticamente nell'apposita
	sezione della pagina "about".
\end{description}