\hypertarget{functions}{%
\section{Functions}\label{functions}}

\begin{description}
\item[{ \protect\hyperlink{openChildWindow}{openChildWindow(pageUrl,
{[}windowIcon{]})} }]
Apre una finestra ``figlia'' e modale.
\item[{ \protect\hyperlink{createWindow}{createWindow()} }]
Crea la finestra principale.
\item[{ \protect\hyperlink{promptModal}{promptModal(parentWindow,
options, callback)} }]
Creazione della finestra di dialogo.
\item[{ \protect\hyperlink{openMobileNavigation}{openMobileNavigation()}
}]
Apre la navbar in modalità ``mobile''. Questa funzione è mantenuta solo
per consentire un eventuale eccessivo ridimensionamento della finestra.
\item[{ \protect\hyperlink{drop}{drop(name, {[}defaultLinkClass{]})} }]
Permette, alla pressione di un bottone, di aprire una sottolista della
navbar.
\item[{ \protect\hyperlink{setLinks}{setLinks(links)} }]
Cambia i link e i nomi dell'argomento precedente e quello successivo a
quello attuale
\item[{ \protect\hyperlink{initialize}{initialize(initial, base)} }]
Funzione che, al caricamento della pagina, si occupa di impostare il
numero di tag section presenti all'interno della pagina nella memoria
locale del browser, di impostare come sezione visibile corrente la prima
(sempre all'interno della memoria locale) e di nascondere tutti i tag
section successivi al primo.
\item[{ \protect\hyperlink{changeTopic}{changeTopic(topicName,
{[}base{]})} }]
Cambia l'argomento correntemente mostrato.
\item[{ \protect\hyperlink{initializeQuiz}{initializeQuiz()} }]
Inizializza la pagina del quiz.
\item[{ \protect\hyperlink{changeQuizSlide}{changeQuizSlide(finalSlide)}
}]
Cambia la slide del quiz attualmente mostrata.
\item[{ \protect\hyperlink{checkQuiz}{checkQuiz()} }]
Calcola i risultati del quiz.
\item[{ \protect\hyperlink{playStopAudio}{playStopAudio(audioTagId,
buttonRef, stopButtonId)} }]
Permette di avviare, mettere in pausa o stoppare un audio.
\item[{ \protect\hyperlink{showExitDialog}{showExitDialog()} }]
Mostra il dialogo di richiesta di conferma di uscita.
\item[{ \protect\hyperlink{openInBrowser}{openInBrowser(link)} }]
Apre un link nel browser predefinito.
\item[{ \protect\hyperlink{openModal}{openModal(content,
{[}windowIcon{]})} }]
Apre una finestra modale mostrante il contenuto richiesto.
\item[{
\protect\hyperlink{openOnKeyboardShortcut}{openOnKeyboardShortcut(shortcut,
content, {[}openAsModal{]})} }]
Apre, tramite una shortcut da tastiera, una finestra mostrante il
contenuto richiesto.
\end{description}

\protect\hypertarget{openChildWindow}{}{}

\hypertarget{openchildwindowpageurl-windowicon}{%
\section{openChildWindow(pageUrl,
{[}windowIcon{]})}\label{openchildwindowpageurl-windowicon}}

Apre una finestra ``figlia'' e modale.

\textbf{Kind}: global function

\begin{longtabu} to \textwidth {X[1,L,m]X[1,L,m]X[1.5,L,m]X[1.5,L,m]}
\toprule
Param & Type & Default & Description\tabularnewline
\midrule
\endhead
pageUrl & \texttt{String} & & L'URL della pagina da aprire (assoluto o
relativo)\tabularnewline
{[}windowIcon{]} & \texttt{String} & \texttt{./assets/icon.ico} &
L'icona della finestra.\tabularnewline
\bottomrule
\end{longtabu}

\protect\hypertarget{createWindow}{}{}

\hypertarget{createwindow}{%
\section{createWindow()}\label{createwindow}}

Crea la finestra principale.

\textbf{Kind}: global function\\
\protect\hypertarget{promptModal}{}{}

\hypertarget{promptmodalparentwindow-options-callback}{%
\section{promptModal(parentWindow, options,
callback)}\label{promptmodalparentwindow-options-callback}}

Creazione della finestra di dialogo.

\textbf{Kind}: global function

\begin{longtabu} to \textwidth {X[1,L,m]X[1,L,m]X[1.5,L,m]}
\toprule
Param & Type & Description\tabularnewline
\midrule
\endhead
parentWindow & \texttt{BrowserWindow} & La finestra
``genitore''\tabularnewline
options & \texttt{Object} & Le opzioni della nuova
finestra\tabularnewline
callback & \texttt{*} & La funzione da richiamare alla chiusura della
finestra\tabularnewline
\bottomrule
\end{longtabu}

\protect\hypertarget{openMobileNavigation}{}{}

\hypertarget{openmobilenavigation}{%
\section{openMobileNavigation()}\label{openmobilenavigation}}

Apre la navbar in modalità ``mobile''. Questa funzione è mantenuta solo
perconsentire un eventuale eccessivo ridimensionamento della finestra.

\textbf{Kind}: global function\\
\protect\hypertarget{drop}{}{}

\hypertarget{dropname-defaultlinkclass}{%
\section{drop(name,
{[}defaultLinkClass{]})}\label{dropname-defaultlinkclass}}

Permette, alla pressione di un bottone, di aprire una sottolistadella
navbar.

\textbf{Kind}: global function

\begin{longtabu} to \textwidth {X[1,L,m]X[1,L,m]X[1.5,L,m]}
\toprule
Param & Type & Description\tabularnewline
\midrule
\endhead
name & \texttt{String} & L'ID della lista che si vuole
controllare\tabularnewline
{[}defaultLinkClass{]} & \texttt{String} & La classe iniziale del
bottone\tabularnewline
\bottomrule
\end{longtabu}

\protect\hypertarget{setLinks}{}{}

\hypertarget{setlinkslinks}{%
\section{setLinks(links)}\label{setlinkslinks}}

Cambia i link e i nomi dell'argomento precedente e quello successivoa
quello attuale

\textbf{Kind}: global function

\begin{longtabu} to \textwidth {X[1,L,m]X[1,L,m]X[1.5,L,m]}
\toprule
\begin{minipage}[b]{0.30\columnwidth}\raggedright
Param\strut
\end{minipage} & \begin{minipage}[b]{0.30\columnwidth}\raggedright
Type\strut
\end{minipage} & \begin{minipage}[b]{0.30\columnwidth}\raggedright
Description\strut
\end{minipage}\tabularnewline
\midrule
\endhead
\begin{minipage}[t]{0.30\columnwidth}\raggedright
links\strut
\end{minipage} & \begin{minipage}[t]{0.30\columnwidth}\raggedright
\texttt{Object}\strut
\end{minipage} & \begin{minipage}[t]{0.30\columnwidth}\raggedright
Le nuove impostazioni e link\strut
\end{minipage}\tabularnewline
\begin{minipage}[t]{0.30\columnwidth}\raggedright
links.previous\strut
\end{minipage} & \begin{minipage}[t]{0.30\columnwidth}\raggedright
\texttt{String}\strut
\end{minipage} & \begin{minipage}[t]{0.30\columnwidth}\raggedright
Il nome della pagina precedente\strut
\end{minipage}\tabularnewline
\begin{minipage}[t]{0.30\columnwidth}\raggedright
links.previousLink\strut
\end{minipage} & \begin{minipage}[t]{0.30\columnwidth}\raggedright
\texttt{String}\strut
\end{minipage} & \begin{minipage}[t]{0.30\columnwidth}\raggedright
Il link della pagina precedente (il nome del file \emph{senza}
l'estensione)\strut
\end{minipage}\tabularnewline
\begin{minipage}[t]{0.30\columnwidth}\raggedright
links.next\strut
\end{minipage} & \begin{minipage}[t]{0.30\columnwidth}\raggedright
\texttt{String}\strut
\end{minipage} & \begin{minipage}[t]{0.30\columnwidth}\raggedright
Il nome della pagina successiva\strut
\end{minipage}\tabularnewline
\begin{minipage}[t]{0.30\columnwidth}\raggedright
links.nextLink\strut
\end{minipage} & \begin{minipage}[t]{0.30\columnwidth}\raggedright
\texttt{String}\strut
\end{minipage} & \begin{minipage}[t]{0.30\columnwidth}\raggedright
Il link della pagina successiva (il nome del file \emph{senza}
l'estensione)\strut
\end{minipage}\tabularnewline
\bottomrule
\end{longtabu}

\protect\hypertarget{initialize}{}{}

\hypertarget{initializeinitial-base}{%
\section{initialize(initial, base)}\label{initializeinitial-base}}

Funzione che, al caricamento della pagina, si occupa di impostare il
numero di tag section presenti all'interno della pagina nella memoria
locale del browser, diimpostare come sezione visibile corrente la prima
(sempre all'interno della memoria locale)e di nascondere tutti i tag
section successivi al primo.

\textbf{Kind}: global function

\begin{longtabu} to \textwidth {X[1,L,m]X[1,L,m]X[1.5,L,m]X[1.5,L,m]}
\toprule
\begin{minipage}[b]{0.22\columnwidth}\raggedright
Param\strut
\end{minipage} & \begin{minipage}[b]{0.22\columnwidth}\raggedright
Type\strut
\end{minipage} & \begin{minipage}[b]{0.22\columnwidth}\raggedright
Default\strut
\end{minipage} & \begin{minipage}[b]{0.22\columnwidth}\raggedright
Description\strut
\end{minipage}\tabularnewline
\midrule
\endhead
\begin{minipage}[t]{0.22\columnwidth}\raggedright
initial\strut
\end{minipage} & \begin{minipage}[t]{0.22\columnwidth}\raggedright
\texttt{String}\strut
\end{minipage} & \begin{minipage}[t]{0.22\columnwidth}\raggedright
\strut
\end{minipage} & \begin{minipage}[t]{0.22\columnwidth}\raggedright
Il primo argomento\strut
\end{minipage}\tabularnewline
\begin{minipage}[t]{0.22\columnwidth}\raggedright
base\strut
\end{minipage} & \begin{minipage}[t]{0.22\columnwidth}\raggedright
\texttt{String}\strut
\end{minipage} & \begin{minipage}[t]{0.22\columnwidth}\raggedright
\texttt{./}\strut
\end{minipage} & \begin{minipage}[t]{0.22\columnwidth}\raggedright
La cartella in cui sono situati i file degli argomenti (default:
\texttt{./})\strut
\end{minipage}\tabularnewline
\bottomrule
\end{longtabu}

\protect\hypertarget{initialize..changeSlide}{}{}

\hypertarget{initializechangeslideslide}{%
\subsection{initialize\textasciitilde{}changeSlide(slide)}\label{initializechangeslideslide}}

La funzione, in base al valore assunto da slide (true/false) cambia la
sezione corrente in quella precedente (in caso di slide = false)o in
quella successiva (in caso di slide = true). Inoltre si occupa di
aggiornare il numero della slide corrente nella memoria temporaneadel
browser. Inoltre, in base al numero di slide, si occupa di
renderevisibili (o nascondere) i relativi pulsanti di spostamento(avanti
con id next, indietro con id back e quiz con id quiz).

\textbf{Kind}: inner method of
\protect\hyperlink{initialize}{\texttt{initialize}}

\begin{longtabu} to \textwidth {X[1,L,m]X[1,L,m]X[1.5,L,m]}
\toprule
\begin{minipage}[b]{0.30\columnwidth}\raggedright
Param\strut
\end{minipage} & \begin{minipage}[b]{0.30\columnwidth}\raggedright
Type\strut
\end{minipage} & \begin{minipage}[b]{0.30\columnwidth}\raggedright
Description\strut
\end{minipage}\tabularnewline
\midrule
\endhead
\begin{minipage}[t]{0.30\columnwidth}\raggedright
slide\strut
\end{minipage} & \begin{minipage}[t]{0.30\columnwidth}\raggedright
\texttt{boolean}\strut
\end{minipage} & \begin{minipage}[t]{0.30\columnwidth}\raggedright
se è \texttt{true}, avanza di una slide, altrimenti indietreggia di una
slide.\strut
\end{minipage}\tabularnewline
\bottomrule
\end{longtabu}

\protect\hypertarget{changeTopic}{}{}

\hypertarget{changetopictopicname-base}{%
\section{changeTopic(topicName,
{[}base{]})}\label{changetopictopicname-base}}

Cambia l'argomento correntemente mostrato.

\textbf{Kind}: global function

\begin{longtabu} to \textwidth {X[1,L,m]X[1,L,m]X[1.5,L,m]X[1.5,L,m]}
\toprule
Param & Type & Default & Description\tabularnewline
\midrule
\endhead
topicName & \texttt{String} & & Il prossimo argomento\tabularnewline
{[}base{]} & \texttt{String} & \texttt{./} & La cartella in cui è
situato il file dell'argomento\tabularnewline
\bottomrule
\end{longtabu}

\protect\hypertarget{initializeQuiz}{}{}

\hypertarget{initializequiz}{%
\section{initializeQuiz()}\label{initializequiz}}

Inizializza la pagina del quiz.

\textbf{Kind}: global function\\
\protect\hypertarget{changeQuizSlide}{}{}

\hypertarget{changequizslidefinalslide}{%
\section{changeQuizSlide(finalSlide)}\label{changequizslidefinalslide}}

Cambia la slide del quiz attualmente mostrata.

\textbf{Kind}: global function

\begin{longtabu} to \textwidth {X[1,L,m]X[1,L,m]X[1.5,L,m]}
\toprule
\begin{minipage}[b]{0.30\columnwidth}\raggedright
Param\strut
\end{minipage} & \begin{minipage}[b]{0.30\columnwidth}\raggedright
Type\strut
\end{minipage} & \begin{minipage}[b]{0.30\columnwidth}\raggedright
Description\strut
\end{minipage}\tabularnewline
\midrule
\endhead
\begin{minipage}[t]{0.30\columnwidth}\raggedright
finalSlide\strut
\end{minipage} & \begin{minipage}[t]{0.30\columnwidth}\raggedright
\texttt{number}\strut
\end{minipage} & \begin{minipage}[t]{0.30\columnwidth}\raggedright
La slide da aprire in seguito alla richiesta di variazione della slide.
Tale valore deve essere compreso nell'intervallo \texttt{{[}0,\ n{]}},
dove \texttt{n} è il numero di slide presenti nella pagina.\strut
\end{minipage}\tabularnewline
\bottomrule
\end{longtabu}

\protect\hypertarget{checkQuiz}{}{}

\hypertarget{checkquiz}{%
\section{checkQuiz()}\label{checkquiz}}

Calcola i risultati del quiz.

\textbf{Kind}: global function\\
\protect\hypertarget{playStopAudio}{}{}

\hypertarget{playstopaudioaudiotagid-buttonref-stopbuttonid}{%
\section{playStopAudio(audioTagId, buttonRef,
stopButtonId)}\label{playstopaudioaudiotagid-buttonref-stopbuttonid}}

Permette di avviare, mettere in pausa o stoppare un audio.

\textbf{Kind}: global function

\begin{longtabu} to \textwidth {X[1,L,m]X[1,L,m]X[1.5,L,m]}
\toprule
\begin{minipage}[b]{0.30\columnwidth}\raggedright
Param\strut
\end{minipage} & \begin{minipage}[b]{0.30\columnwidth}\raggedright
Type\strut
\end{minipage} & \begin{minipage}[b]{0.30\columnwidth}\raggedright
Description\strut
\end{minipage}\tabularnewline
\midrule
\endhead
\begin{minipage}[t]{0.30\columnwidth}\raggedright
audioTagId\strut
\end{minipage} & \begin{minipage}[t]{0.30\columnwidth}\raggedright
\texttt{String}\strut
\end{minipage} & \begin{minipage}[t]{0.30\columnwidth}\raggedright
L'ID dell'elemento \texttt{\textless{}audio\textgreater{}} da
controllare\strut
\end{minipage}\tabularnewline
\begin{minipage}[t]{0.30\columnwidth}\raggedright
buttonRef\strut
\end{minipage} & \begin{minipage}[t]{0.30\columnwidth}\raggedright
\texttt{HTMLElement}\strut
\end{minipage} & \begin{minipage}[t]{0.30\columnwidth}\raggedright
Un riferimento al bottone che richiama questa funzione\strut
\end{minipage}\tabularnewline
\begin{minipage}[t]{0.30\columnwidth}\raggedright
stopButtonId\strut
\end{minipage} & \begin{minipage}[t]{0.30\columnwidth}\raggedright
\texttt{String}\strut
\end{minipage} & \begin{minipage}[t]{0.30\columnwidth}\raggedright
L'ID del bottone di Stop.\strut
\end{minipage}\tabularnewline
\bottomrule
\end{longtabu}

\protect\hypertarget{showExitDialog}{}{}

\hypertarget{showexitdialog}{%
\section{showExitDialog()}\label{showexitdialog}}

Mostra il dialogo di richiesta di conferma di uscita.

\textbf{Kind}: global function\\
\protect\hypertarget{openInBrowser}{}{}

\hypertarget{openinbrowserlink}{%
\section{openInBrowser(link)}\label{openinbrowserlink}}

Apre un link nel browser predefinito.

\textbf{Kind}: global function

\begin{longtabu} to \textwidth {X[1,L,m]X[1,L,m]X[1.5,L,m]}
\toprule
Param & Type & Description\tabularnewline
\midrule
\endhead
link & \texttt{String} & Il link da aprire\tabularnewline
\bottomrule
\end{longtabu}

\protect\hypertarget{openModal}{}{}

\hypertarget{openmodalcontent-windowicon}{%
\section{openModal(content,
{[}windowIcon{]})}\label{openmodalcontent-windowicon}}

Apre una finestra modale mostrante il contenuto richiesto.

\textbf{Kind}: global function

\begin{longtabu} to \textwidth {X[1,L,m]X[1,L,m]X[1.5,L,m]X[1.5,L,m]}
\toprule
Param & Type & Default & Description\tabularnewline
\midrule
\endhead
content & \texttt{String} & & Il link (assoluto o relativo) da
aprire\tabularnewline
{[}windowIcon{]} & \texttt{String} & \texttt{./assets/icon.ico} &
L'icona della finestra modale\tabularnewline
\bottomrule
\end{longtabu}

\protect\hypertarget{openOnKeyboardShortcut}{}{}

\hypertarget{openonkeyboardshortcutshortcut-content-openasmodal}{%
\section{openOnKeyboardShortcut(shortcut, content,
{[}openAsModal{]})}\label{openonkeyboardshortcutshortcut-content-openasmodal}}

Apre, tramite una shortcut da tastiera,una finestra mostrante il
contenuto richiesto.

\textbf{Kind}: global function

\begin{longtabu} to \textwidth {X[1,L,m]X[1,L,m]X[1.5,L,m]X[1.5,L,m]}
\toprule
\begin{minipage}[b]{0.22\columnwidth}\raggedright
Param\strut
\end{minipage} & \begin{minipage}[b]{0.22\columnwidth}\raggedright
Type\strut
\end{minipage} & \begin{minipage}[b]{0.22\columnwidth}\raggedright
Default\strut
\end{minipage} & \begin{minipage}[b]{0.22\columnwidth}\raggedright
Description\strut
\end{minipage}\tabularnewline
\midrule
\endhead
\begin{minipage}[t]{0.22\columnwidth}\raggedright
shortcut\strut
\end{minipage} & \begin{minipage}[t]{0.22\columnwidth}\raggedright
\texttt{String}\strut
\end{minipage} & \begin{minipage}[t]{0.22\columnwidth}\raggedright
\strut
\end{minipage} & \begin{minipage}[t]{0.22\columnwidth}\raggedright
La shortcut da utilizzare\strut
\end{minipage}\tabularnewline
\begin{minipage}[t]{0.22\columnwidth}\raggedright
content\strut
\end{minipage} & \begin{minipage}[t]{0.22\columnwidth}\raggedright
\texttt{String}\strut
\end{minipage} & \begin{minipage}[t]{0.22\columnwidth}\raggedright
\strut
\end{minipage} & \begin{minipage}[t]{0.22\columnwidth}\raggedright
Il link (assoluto o relativo) da aprire\strut
\end{minipage}\tabularnewline
\begin{minipage}[t]{0.22\columnwidth}\raggedright
{[}openAsModal{]}\strut
\end{minipage} & \begin{minipage}[t]{0.22\columnwidth}\raggedright
\texttt{boolean}\strut
\end{minipage} & \begin{minipage}[t]{0.22\columnwidth}\raggedright
\texttt{false}\strut
\end{minipage} & \begin{minipage}[t]{0.22\columnwidth}\raggedright
Se è \texttt{true}, la finestra sarà aperta come modale, altrimenti sarà
aperta nella stessa finestra.\strut
\end{minipage}\tabularnewline
\bottomrule
\end{longtabu}
