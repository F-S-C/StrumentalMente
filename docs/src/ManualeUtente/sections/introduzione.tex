\chapter{Introduzione}

\section{Introduzione al sistema}

StrumentalMente è una applicazione \emph{desktop} sviluppata dal
\emph{team} FSC (\emph{Five Students of Computer Science)} per
avvicinare diverse tipologie di utenti allo straordinario mondo della
musica.

\subsection{Gli obiettivi}

Strumentalmente ha l'obiettivo di fungere da veicolo per entrare a
contatto con la musica. Per questo motivo, presenta sia dei concetti
teorico/pratici preliminari che avanzati, per accontentare tutte le
tipologie di utente.

\section{La struttura del manuale}

\begin{table}[H]
	\centering
	\begin{tabular}{@{}l@{}}
		\toprule
		\multicolumn{1}{c}{\textsc{\textbf{Legenda}}}\\
		\midrule
		{\Large\faInfoCircle{}}~~~Informazione fondamentale\\
		\midrule
		{\Large\faKeyboardO{}}~~~Scorciatoia da tastiera\\
		\midrule
		{\Large\faWarning{}}~~~Nota bene\\
		\bottomrule
	\end{tabular}
	\caption{Lista delle icone utilizzate nel manuale}
	\label{tab:icons}
\end{table}

Questo manuale ha il compito di presentare il funzionamento di
StrumentalMente, nonché quello di aiutare l'utente nel processo di
installazione, fruizione e (eventualmente) correzione degli errori. Nel
farlo, si utilizzeranno alcune notazioni, che sono riportate nella
Tabella \ref{tab:icons}: vi saranno nel corso di queste manuale alcune sezioni che saranno segnalate con un'icona che rappresenta il significato delle stesse. 

Il presente manuale è, quindi, suddiviso in tre parti principali in cui
sarà possibile trovare assistenza per i processi precedentemente
elencati, ovverosia il processo di installazione, di fruizione e di
gestione degli eventuali errori.

\section{Licenze e copyright}

StrumentalMente è un progetto \emph{open source} (il codice sorgente è
disponibile gratuitamente \emph{online} sulla piattaforma GitHub,
all'indirizzo
\url{https://github.com/F-S-C/StrumentalMente}).
È rilasciato sotto la \emph{Licenza Apache 2.0} (per maggiori
informazioni si legga il relativo file \texttt{LICENSE} presente all'interno
della \emph{repository}). Il progetto è stato sviluppato ed è mantenuto
dal \emph{team} FSC, che ne detiene i diritti di \emph{copyright}.