%
% file: localoperator.tex
% author: Victor Brena
% description: Briefly describes properties of the local operator.
%

\chapter{Crediti}
\section{Il team di sviluppo}

FSC, il team che ha sviluppato StrumentalMente, è composto da:
\begin{description}
    \item[Alessandro Annese:] \textit{graphic designer}, esperto di grafica e multimedialità, \textit{videomaker} e sviluppatore
    \item[Davide De Salvo:] musicista, esperto del dominio, \textit{web editor}, supporto al \textit{designer} e sviluppatore
    \item[Andrea Esposito:] \textit{project manager}, referente del \textit{team}, \textit{technical writer}, fotografo e sviluppatore
    \item[Graziano Montanaro:] \textit{web editor}, supporto al \textit{copywriter}, \textit{data entry} e sviluppatore
    \item[Regina Zaccaria:] \textit{copywriter}, \textit{ghostwriter}, gestore dei contenuti, organizzatrice degli eventi e sviluppatrice
\end{description}

\section{Ringraziamenti}
Un sentito ringraziamento va all'Accademia Musicale \textit{Francisco Tàrrega} di Taranto che ha messo a disposizione del team degli istruttori di musica. Si ringrazia, in particolare:
\begin{itemize}
    \item Andrea Manco, istruttore teorico
    \item William Marino, istruttore di basso
    \item Giovanni Pagliaro, istruttore di chitarra
    \item Marcello Nisi, istruttore di batteria
    \item Marco Amati, istruttore di pianoforte
\end{itemize}
\subsection{I beta tester}
Un doveroso ringraziamento è diretto a tutti coloro che hanno reso possibile la realizzazione di StrumentalMente aiutando il team nel testing dell'applicazione. Si ringraziano, quindi: \csvreader[head to column names,late after line={{,}\ }, late after last line=.]{../beta-testers.csv}{}%
{\nome\ \cognome}