%\input{setup/includes.tex}

\usepackage[italian]{babel}
\frenchspacing
\usepackage[sorting=nty]{biblatex}
\usepackage[utf8]{inputenc}
\usepackage{amsmath}
\usepackage{graphicx}
\graphicspath{{images/}}
\usepackage{parskip}
\usepackage{fancyhdr}
\usepackage{array}
\usepackage{multirow}
\usepackage{tabularx}
\usepackage[table,xcdraw]{xcolor}
\usepackage{lipsum}
\usepackage{cancel}
\usepackage{longtable}
\usepackage{lettrine} % The lettrine is the first enlarged letter at the beginning of the text
\usepackage{vmargin}
\usepackage{hyperref}
\usepackage{wrapfig}
\usepackage{eurosym}
\hypersetup{breaklinks=true}
\usepackage{url}
\usepackage[anythingbreaks]{breakurl}
\usepackage{bold-extra}
\usepackage{booktabs}
\usepackage{tabu}
\usepackage{textcomp}
\usepackage{calc}
\usepackage{subcaption}

\usepackage{listings}
\usepackage{color}
\usepackage{tcolorbox}

\usepackage{titlesec}

\usepackage{caption}
\usepackage[framed,amsmath,thmmarks]{ntheorem}
\usepackage{tabularx}
\usepackage{float}

% ------

%\input{setup/setup.tex}

% DECOMMENT FOR EQUAL MARGINS
%\setmarginsrb{3 cm}{2.5 cm}{3 cm}{2.5 cm}{1 cm}{1.5 cm}{1 cm}{1.5 cm}

%%%%%%%%%%%%%%%%%%%%%%%%%%%%%%%%%%%%%%%%%%%%%%%%
% An example environment
%%%%%%%%%%%%%%%%%%%%%%%%%%%%%%%%%%%%%%%%%%%%%%%%
\captionsetup{%
	font=footnotesize,% set font size to footnotesize
	labelfont=bf % bold label (e.g., Figure 3.2) font
}

\theoremheaderfont{\normalfont\bfseries}
\theorembodyfont{\normalfont}
\theoremstyle{break}
\def\theoremframecommand{{\color{gray!50}\vrule width 5pt \hspace{5pt}}}
\newshadedtheorem{exa}{\iflanguage{italian}{Esempio}{Example}}[chapter]
\newenvironment{example}[1]{%
	\begin{exa}[#1]
	}{%
	\end{exa}
}

\definecolor{dkgreen}{rgb}{0,0.6,0}
\definecolor{gray}{rgb}{0.5,0.5,0.5}
\definecolor{mauve}{rgb}{0.58,0,0.82}

\definecolor{mainColorDark}{HTML}{E55100} % Inizialmente 003A8F

\lstset{
	language=C,
	showstringspaces=false,
	columns=flexible,
	basicstyle={\small\ttfamily},
	numbers=none,
	numberstyle=\tiny\color{gray},
	keywordstyle=\color{blue},
	commentstyle=\color{dkgreen},
	stringstyle=\color{mauve},
	breaklines=true,
	breakatwhitespace=true,
	tabsize=3
}


\titleformat{\chapter}[hang] 
{\normalfont\huge\bfseries}{\chaptertitlename\ \thechapter}{20pt}{} 
\makeatletter
\renewcommand{\@chapapp}{}% Not necessary...
\newenvironment{chapquote}[2][2em]
{
	\setlength{\@tempdima}{#1}%
	\def\chapquote@author{#2}%
	\parshape 1 \@tempdima \dimexpr\textwidth-2\@tempdima\relax%
	\itshape
}
{
	\par\normalfont\hfill--\ \chapquote@author\hspace*{\@tempdima}\par\bigskip
}
\makeatother

\titlespacing*{\chapter} {0pt}{-15pt}{2.3ex plus .2ex}


\renewcommand\tabularxcolumn[1]{m{#1}}
\renewcommand{\arraystretch}{2}
\tabulinesep =^3mm_3mm
\newcolumntype{M}[1]{>{\centering\arraybackslash}m{#1}}

% ------


\makeatletter
\let\thetitle\@title
\let\theauthor\@author
\let\thedate\@date
\makeatother

%\input{setup/setHeaderFooter.tex}

%Decommenta se usi \leftmark
\renewcommand{\chaptermark}[1]{\markboth{{#1}}{}}

\fancypagestyle{plain}
{
	\fancyhead{}
	\fancyfoot{}
	\renewcommand{\headrulewidth}{1pt}
	\renewcommand{\footrulewidth}{1pt}
	\fancyfoot[RO,LE]{\thepage}
	%	\fancyhead[RO,LE]{\theteam}
	\fancyhead[LO,RE]{\textbf{\ProjectTitle}} %Oppure 
	\fancyhead[LE,RO]{\nouppercase{\leftmark}}
}

\pagestyle{fancy}
\fancyhead{}
\fancyfoot{}
\renewcommand{\headrulewidth}{1pt}
\renewcommand{\footrulewidth}{1pt}
\fancyfoot[RO,LE]{\thepage}
%\fancyhead[RO,LE]{\theteam}
\fancyhead[LO,RE]{\textbf{\ProjectTitle}} %Oppure 
\fancyhead[LE,RO]{\nouppercase{\leftmark}}

\sloppy
% ------