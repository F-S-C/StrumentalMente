\chapter*{Prefazione}
\section*{Il team}

\begin{quote}
\flushright \emph{``Il nostro obiettivo è quello di portare a termine questo progetto e i successivi nel migliore dei modi, nonché quello di creare una squadra forte e duratura che possa sopravvivere al termine del nostro percorso di studi.''}

--- Il team F.S.C.
\end{quote}

Il team di sviluppo del progetto è il team ``\textbf{F.S.C.}'' (\emph{Five
Students of Computer Science}). È un team composto da cinque studenti iscritti
per l'anno accademico 2018/19 al secondo anno del Corso di Laurea in
\emph{Informatica e Comunicazione Digitale} (\emph{I.C.D.}) dell'Università
degli Studi di Bari. 

Gli studenti che compongono il gruppo, così come indicati sul frontespizio di
questo documento, sono: Alessandro \textbf{Annese}, Davide \textbf{De Salvo},
\emph{Andrea \textbf{Esposito}}, Graziano \textbf{Montanaro}, Regina
\textbf{Zaccaria}. Il referente del gruppo per il presente progetto è A.
Esposito.

\section*{Il progetto}
Il progetto ``\ProjectTitle{}'' nasce come progetto d'esame per il corso di
\emph{Progettazione e Produzione Multimediale} del secondo anno del corso di
laurea in I.C.D. dell'Università di Bari.

Le fasi dello sviluppo di questo sistema seguono il \textbf{modello di Alessi \&
Trollip} per la progettazione e lo sviluppo di un'applicazione multimediale.

Il presente testo è un'unione dei vari documenti prodotti durante la
progettazione e lo sviluppo di ``\ProjectTitle{}'', ovverosia quello di
pianificazione, di progettazione e quello di test. Si è scelto di presentarli in
un unico documento per favorire il processo di stampa, nonché per mostrare tutte
le fasi dello sviluppo in modo organico e completo.

Tutto i file multimediali, il codice sorgente nonché questo documento e quelli
sopraccitati sono disponibili per la consultazione su GitHub al seguente link.

\begin{center}
	\url{https://github.com/F-S-C/StrumentalMente}
\end{center}