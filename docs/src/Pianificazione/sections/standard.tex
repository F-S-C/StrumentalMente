\section{Manuale di stile}
Il sistema avrà un \emph{look} moderno e accattivante che possa enfatizzare la
sua semplicità di utilizzo. A tale fine si utilizzeranno dei font \emph{sans
serif} (\texttt{Montserrat} e \texttt{Raleway}).

\subsection{I colori}
I colori predominanti saranno una miscela di colori neutri e caldi, ovverosia
bianco, grigio scuro e rosso-arancio (il colore associato alla nota Do). Si
utilizzeranno le linee guida dettate dal \emph{Material Design} di \emph{Google}
per orchestrarli al meglio e per raggiungere l'obiettivo di un \emph{look}
semplice, moderno e accattivante.

\subsection{La navigazione}
Il sistema sarà navigabile utilizzando dei \emph{link} e dei bottoni, che
rispetteranno le linee guida del Material Design. In base al colore dello sfondo
su cui i bottoni saranno inseriti, tali bottoni possono essere trasparenti con
testo rosso-arancio (su sfondi chiari) o viceversa (su sfondi scuri).

\subsection{Le pagine} 
Ogni pagina dell'applicazione sarà caratterizzata da un aspetto simile alle
altre, con delle piccole differenze in base alla categoria di pagina e alla
tipologia di informazioni che conterrà (per maggiori informazioni si veda la
sezione \ref{sec:contenuti} sui contenuti).

\section{I contenuti}\label{sec:contenuti}
Il sistema sarà diviso in due unità: teoria e pratica. Per poter comprendere ciò
che sarà presentato nell'unità della pratica, è necessario aver compreso tutto
ciò che la prima sezione dell'unità della teoria presenta.

Saranno inoltre introdotti dei test di autovalutazione per valutare le
competenze acquisite durante l'uso del sistema. Il superamento di tali test non
è un requisito per la navigazione delle varie parti dell'applicazione, tuttavia
è fortemente consigliato per assicurarsi una migliore comprensione degli
argomenti avanzati.

\subsection{Unità 1: la teoria}
Questa unità è suddivisa in due sezioni:
\begin{enumerate}
	\item Teoria musicale di livello basico
	\item Teoria musicale di livello intermedio/avanzato
\end{enumerate}

\paragraph{Teoria musicale di livello basico} In questa sezione si presenteranno
tutti i concetti basilari senza i quali l'utente non può utilizzare a dovere il
sistema. Sarà fornito all'utente un lessico basilare che gli possa permettere di
comprendere ciò che \ProjectTitle{} (e gli approfondimenti suggeriti al suo
interno) presenta e descrive. \paragraph{Teoria musicale di livello
intermedio/avanzato} In questa sezione verranno approfonditi i concetti
introdotti nella sezione precedente con uno sguardo meno rivolto alla pratica.
Sarà una sezione di approfondimento che mirerà ad arricchire il vocabolario
tecnico che l'utente, tramite la precedente sezione, ha iniziato a costruire.

\subsection{Unità 2: la pratica}
Quest'unità presenterà le modalità d'uso, le componenti e una lista di tecniche
o accordi per vari strumenti. Gli strumenti che il team ha selezionato dopo
un'iniziale fase di brainstorming sono i seguenti:
\begin{itemize}
	\item Basso
	\item Batteria
	\item Chitarra
	\item Pianoforte/Tastiera
\end{itemize}
Per comprendere appieno le nozioni presentate in quest'unità è richiesto un
vocabolario tecnico minimo, che è possibile acquisire completando la prima
sezione della prima unità.

Si vuole sottolineare che l'applicazione \ProjectTitle{} è un'applicazione che
può essere arricchita dopo il rilascio, aggiungendo eventualmente altri
strumenti alla precedente lista.

\subsection{I test di autovalutazione}
Sono previsti vari test di autovalutazione:
\begin{itemize}
	\item Uno iniziale per permettere un'autovalutazione delle competenze
	iniziali
	\item Uno per la prima sezione dell'unità teorica
	\item Uno per la seconda sezione dell'unità teorica
	\item Uno per ogni strumento presentato all'interno dell'unità pratica (in
	base allo strumento che è scelto dall'utente sarà selezionato quello
	corrispondente)
\end{itemize}
Benché il sistema sia un sistema di \emph{e-learning}, non sarà obbligatorio
completare i vari test per proseguire nella navigazione tra le sezioni. Questa
decisione del team è stata presa in seguito a delle considerazioni
sull'usabilità dell'applicazione da parte di utenti che visitano il sistema più
volte: poiché il sistema può essere utilizzato come supporto allo studio da
altre fonti, deve essere fornita la possibilità all'utente di saltare
direttamente alle nozioni a cui è interessato, senza dover necessariamente
completare una serie di test autovalutativi.

\subsection*{Approvazione del committente}
I contenuti presentati hanno ricevuto l'approvazione da parte del committente il
28 novembre 2018.