La seguente tabella (Tabella \ref{tab:costi}) contiene una lista delle attività da completare per portare al termine il progetto. Tale lista è emersa durante una discussione del \textit{team}.

Oltre a ogni attività, sono elencate il numero di ore che si prevede siano necessarie a portare al termine le stesse.

\begin{tabularx}{\textwidth}{|m{5cm}|X|M{2cm}|}
	\caption{Costi previsti in ore di lavoro}
	\label{tab:costi}\\
	\hline
	\rowcolor{mainColorDark} 
	{\color[HTML]{FFFFFF} \textbf{\textsc{Fasi della produzione}}}                                                           & {\color[HTML]{FFFFFF} \hspace*{\fill}\textbf{\textsc{Attività}}}\hspace*{\fill}              & {\color[HTML]{FFFFFF} \textbf{\textsc{Impegno orario}}} \\ \hline
	%
	\cellcolor{mainColorDark}{\color[HTML]{FFFFFF} }                                                                          & Acquisizione del materiale audio                               & 10                                                      \\ \cline{2-3} 
	\cellcolor{mainColorDark}{\color[HTML]{FFFFFF} }                                                                          & Acquisizione del materiale testuale                            & 10                                                       \\ \cline{2-3} 
	\cellcolor{mainColorDark}{\color[HTML]{FFFFFF} }                                                                          & Acquisizione del materiale video e fotografico                 & 10                                                       \\ \cline{2-3} 
	\cellcolor{mainColorDark}{\color[HTML]{FFFFFF} }                                                                          & Acquisizione del materiale di supporto (tabelle, schede, ecc.) & 4                                                       \\ \cline{2-3}
        \cellcolor{mainColorDark}{\color[HTML]{FFFFFF} }                                                                      & Progettazione dei test di autovalutazione                      & 6                                                        \\ \cline{2-3}  
	\multirow{-6}{4cm}{\cellcolor{mainColorDark}{\color[HTML]{FFFFFF} \textbf{Acquisizione del materiale}}}                   & \textbf{\textsc{Totale}}                                       & \textbf{40}                                             \\ \hline
	\cellcolor{mainColorDark}{\color[HTML]{FFFFFF} }                                                                          & Stesura di un inventario del materiale d'acquisto              & 3                                                       \\ \cline{2-3} 
	\cellcolor{mainColorDark}{\color[HTML]{FFFFFF} }                                                                          & Revisione e correzione del materiale acquisito                 & 5                                                       \\ \cline{2-3} 
	\multirow{-3}{4cm}{\cellcolor{mainColorDark}{\color[HTML]{FFFFFF} \textbf{Verifica e validazione del materiale acquisito}}} & \textbf{\textsc{Totale}}                                     & \textbf{8}                                              \\ \hline
	\cellcolor{mainColorDark}{\color[HTML]{FFFFFF} }                                                                          & Sviluppo degli standard comunicativi                           & 5                                                         \\ \cline{2-3} 
	\cellcolor{mainColorDark}{\color[HTML]{FFFFFF} }                                                                          & Realizzazione della barra di navigazione                       & 5                                                        \\ \cline{2-3} 
	\cellcolor{mainColorDark}{\color[HTML]{FFFFFF} }                                                                          & Realizzazione delle interfacce grafiche                        & 10                                                        \\ \cline{2-3} 
	\multirow{-4}{4cm}{\cellcolor{mainColorDark}{\color[HTML]{FFFFFF} \textbf{Definizione dell'interfaccia utente}}}            & \textbf{\textsc{Totale}}                                     & \textbf{20}                                               \\ \hline
	\cellcolor{mainColorDark}{\color[HTML]{FFFFFF} }                                                                          & Realizzazione delle pagine                                     & 30                                                        \\ \cline{2-3} 
	\cellcolor{mainColorDark}{\color[HTML]{FFFFFF} }                                                                          & Realizzazione delle interazioni tra le pagine                  & 12                                                        \\ \cline{2-3} 
	\cellcolor{mainColorDark}{\color[HTML]{FFFFFF} }                                                                          & Realizzazione e ottimizzazione dell'interazione                & 8                                                        \\ \cline{2-3} 
	\cellcolor{mainColorDark}{\color[HTML]{FFFFFF} }                                                                          & Realizzazione dei manuali                                      & 4                                                        \\ \cline{2-3} 
	\cellcolor{mainColorDark}{\color[HTML]{FFFFFF} }                                                                          & Produzione della versione $\alpha$                             & 2                                                        \\ \cline{2-3} 
	\multirow{-6}{4cm}{\cellcolor{mainColorDark}{\color[HTML]{FFFFFF} \textbf{Sviluppo}}}                                       & \textbf{\textsc{Totale}}                                     & \textbf{56}                                               \\ \hline
	\cellcolor{mainColorDark}{\color[HTML]{FFFFFF} }                                                                          & Alpha test e documento di test                                 & 10                                                        \\ \cline{2-3} 
	\cellcolor{mainColorDark}{\color[HTML]{FFFFFF} }                                                                          & Revisione del software                                         & 10                                                        \\ \cline{2-3} 
	\cellcolor{mainColorDark}{\color[HTML]{FFFFFF} }                                                                          & Beta test e documento di test                                  & 10                                                        \\ \cline{2-3} 
	\multirow{-4}{4cm}{\cellcolor{mainColorDark}{\color[HTML]{FFFFFF} \textbf{Test}}}                                           & \textbf{\textsc{Totale}}                                     & \textbf{30}                                               \\ \hline
	\cellcolor{mainColorDark}{\color[HTML]{FFFFFF} }                                                                          & Realizzazione copia master                                     & 2                                                        \\ \cline{2-3} 
	\cellcolor{mainColorDark}{\color[HTML]{FFFFFF} }                                                                          & Realizzazione delle copie per sviluppatori e commitente        & 2                                                        \\ \cline{2-3} 
	\multirow{-3}{4cm}{\cellcolor{mainColorDark}{\color[HTML]{FFFFFF} \textbf{Pubblicazione}}}                                  & \textbf{\textsc{Totale}}                                     & \textbf{4}                                               \\ \hline
\end{tabularx}

\section{Documento di pianificazione}
Il presente documento è stato modificato dopo circa una settimana di lavoro per poter includere le percentuali di completamento relative delle varie attività previste.

Le percentuali di completamento presenti in questa tabella (Tabella \ref{tab:docPianificazione}) sono percentuali empiriche basate su un calcolo approssimativo della mole di lavoro compiuta, che è poi stata paragonata alla mole di lavoro prevista per portare al termine una singola attività.
 
\begin{tabularx}{\textwidth}{|X|M{2cm}|M{2cm}|M{2.5cm}|}
	\caption[Costi e percentuali di completamento]{Costi in ore e percentuali di completamento delle attività previste durante la pianificazione}
	\label{tab:docPianificazione}\\
	\hline
	\rowcolor{mainColorDark}
	{\color[HTML]{FFFFFF} \textbf{\footnotesize \textsc{Attività}}} & {\color[HTML]{FFFFFF} \textbf{\footnotesize \textsc{Tempo stimato (ore)}}} & {\color[HTML]{FFFFFF} \textbf{\footnotesize \textsc{Tempo utilizzato (ore)}}} & {\color[HTML]{FFFFFF} \textbf{\footnotesize \textsc{Comple-\allowbreak{}tamento percentuale}}} \\ \hline
	\hline
	\textbf{Acquisizione dei contenuti}               & 40                                                            & x                                                               & x \%                                                                  \\ \hline
	\textbf{Verifica e validazione dei contenuti}     & 8                                                            & x                                                               & x \%                                                                  \\ \hline
	\textbf{Definizione dell'interfaccia utente}      & 20                                                            & x                                                               & x \%                                                                  \\ \hline
	\textbf{Sviluppo}                                 & 56                                                            & x                                                               & x \%                                                                  \\ \hline
	\textbf{Test}                                     & 30                                                            & x                                                               & x \%                                                                  \\ \hline
	\textbf{Pubblicazione}                            & 4                                                            & x                                                               & x \%                                                                  \\ \hline
\end{tabularx}

\section{Risorse}
Di seguito, saranno illustrate tutte le risorse utilizzate per la realizzazione del sistema multimediale.
\subsection{Risorse umane}
La distribuzione del lavoro nel team di progettazione di sviluppo del sistema è stato diviso nel seguente modo:
\begin{itemize}
	\item \textbf{Alessandro Annese:} gestione e produzione degli elementi multimediali del sistema; supporto nella creazione delle pagine del sistema e nella gestione della documentazione.
	\item \textbf{Davide De Salvo:} gestione e produzione degli elementi multimediali del sistema; creazione delle pagine del sistema.
	\item \textbf{Andrea Esposito:} gestione della parte ``\textit{backend}'' dell'applicazione con una speciale attenzione all'utilizzo dei \textit{framework} necessari allo sviluppo dell'applicazione; gestione della documentazione e supporto nella creazione delle pagine del sistema.
	\item \textbf{Graziano Montanaro:} gestione e revisione dei contenuti testuali dell'applicazione; creazione delle pagine del sistema.
	\item \textbf{Regina Zaccaria:} gestione e revisione dei contenuti testuali dell'applicazione; creazione delle pagine del sistema.
\end{itemize}
Ovviamente, la suddivisione dei lavori precedentemente presentata non esclude la possibilità di variazioni successive o di collaborazioni fra membri del team con compiti differenti nella risoluzione di \textit{task} più complessi di quelli attualmente previsti.

\subsection{Risorse informative}\label{sec:risorse-informative}
Tutte le informazioni riguardo gli strumenti musicali e le loro modalità di utilizzo (in senso stretto) sono frutto di studi personali dei singoli componenti del team. 

Le informazioni relative alla teoria musicale e strumentale (nonché le modalità di presentazione delle stesse) saranno reperite da libri di testo o da esperti del settore: 
\begin{wrapfigure}{r}{0.45\textwidth}
	\includegraphics[width=0.9\linewidth]{Logo_Tarrega}
	\label{fig:logo-tarrega}
	\caption{Logo dell'Accademia Musicale Francisco Tàrrega}
\end{wrapfigure}
si prevede una spesa di circa \EUR{100} (cento euro) per poter ricevere supporto nella stesura dei contenuti da parte di esperti del settore (maestri di musica dell'\emph{Accademia musicale \textbf{Francisco Tàrrega}}). 

Si provvederà autonomamente, con un eventuale supporto da parte degli esperti, alla creazione di tutto il materiale multimediale di supporto (foto, audio e video).

\subsection{Risorse applicative}
Nello sviluppo dell'applicazione saranno utilizzati i seguenti applicativi:
\begin{itemize}
	\item \emph{Adobe Photoshop CC} (modifica delle immagini)
	\item \emph{Adobe Illustrator CC} (creazione dei loghi e degli schemi)
	\item \emph{Adobe Premiere CC} (modifica e montaggio video)
	\item \emph{Audacity} (modifica e montaggio audio)
\end{itemize}
Tutti i programmi precedentemente elencati sono o gratuiti o saranno utilizzati nelle loro versioni di prova.

Inoltre, si utilizzerà Git come sistema di controllo delle versioni, in combinazione con la piattaforma GitHub, che sarà usata per condividere i file sorgenti del sistema.

\subsection{Risorse post-produzione}
Per la pubblicazione saranno necessari:
\begin{itemize}
	\item 6 CD-Rom per la creazione del master e delle copie per la distribuzione
	\item Inchiostro e carta per la stampa dei manuali e dei loro prototipi
\end{itemize}
Saranno distribuite due versioni dell'applicativo: una installabile tramite un \textit{installer} per \textit{Windows} e una eseguibile direttamente da CD-Rom.

\section{Stima dei costi}
Si prevedono costi per:
\begin{itemize}
	\item Consulenza di esperti del settore (si veda la sezione \ref{sec:risorse-informative})
	\item Stampa e rilegatura dei manuali e dei documenti di progetto
	\item Produzione e decorazione delle copie su CD-Rom
\end{itemize}

I costi potrebbero variare in base al numero di ore che le componenti del \textit{team} impiegano nella varie fasi dello sviluppo.