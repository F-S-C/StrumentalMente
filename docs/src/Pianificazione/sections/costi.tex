La seguente tabella (Tabella \ref{tab:costi}) contiene una lista delle attività da completare per portare al termine il progetto. Tale lista è emersa durante una discussione del \textit{team}.

Oltre a ogni attività, sono elencate il numero di ore che si prevede siano necessarie a portare al termine le stesse.

\begin{longtabu} to \textwidth {|X[1,L,m]|X[2,L,m]|X[.5,C,m]|}
	\caption{Costi previsti in ore di lavoro.}
	\label{tab:costi}\\
	\hline
	\rowcolor{mainColorDark} 
	{\color[HTML]{FFFFFF} \textbf{\textsc{Fasi della produzione}}}                                                           & {\color[HTML]{FFFFFF} \hspace*{\fill}\textbf{\textsc{Attività}}}\hspace*{\fill}              & {\color[HTML]{FFFFFF} \textbf{\textsc{Impegno orario}}} \\ \hline
	\endfirsthead
	%
	\multicolumn{3}{c}%
	{{\footnotesize\textbf{Tabella \thetable{}:} continuazione della pagina precedente.}} \\
	\hline
	\rowcolor{mainColorDark} 
	{\color[HTML]{FFFFFF} \textbf{\textsc{Fasi della produzione}}}                                                           & {\color[HTML]{FFFFFF} \hspace*{\fill}\textbf{\textsc{Attività}}}\hspace*{\fill}              & {\color[HTML]{FFFFFF} \textbf{\textsc{Impegno orario}}} \\ \hline
	\endhead
	%
	\cellcolor{mainColorDark}{\color[HTML]{FFFFFF} }                                                                          & Acquisizione del materiale audio                               & 10                                                      \\ \cline{2-3} 
	\cellcolor{mainColorDark}{\color[HTML]{FFFFFF} }                                                                          & Acquisizione del materiale testuale                            & 4                                                       \\ \cline{2-3} 
	\cellcolor{mainColorDark}{\color[HTML]{FFFFFF} }                                                                          & Acquisizione del materiale video e fotografico                 & 10                                                       \\ \cline{2-3} 
	\cellcolor{mainColorDark}{\color[HTML]{FFFFFF} }                                                                          & Acquisizione del materiale di supporto (tabelle, schede, ecc.) & x                                                       \\ \cline{2-3}
        \cellcolor{mainColorDark}{\color[HTML]{FFFFFF} }                                                                      & Progettazione dei test di autovalutazione                      & 6                                                        \\ \cline{2-3}  
	\multirow{-6}{4cm}{\cellcolor{mainColorDark}{\color[HTML]{FFFFFF} \textbf{Acquisizione del materiale}}}                   & \textbf{\textsc{Totale}}                                       & \textbf{30+x}                                             \\ \hline
	\cellcolor{mainColorDark}{\color[HTML]{FFFFFF} }                                                                          & Stesura di un inventario del materiale d'acquisto              & 2                                                       \\ \cline{2-3} 
	\cellcolor{mainColorDark}{\color[HTML]{FFFFFF} }                                                                          & Revisione e correzione del materiale acquisito                 & 1                                                       \\ \cline{2-3} 
	\multirow{-3}{4cm}{\cellcolor{mainColorDark}{\color[HTML]{FFFFFF} \textbf{Verifica e validazione del materiale acquisito}}} & \textbf{\textsc{Totale}}                                     & \textbf{3}                                              \\ \hline
	\cellcolor{mainColorDark}{\color[HTML]{FFFFFF} }                                                                          & Sviluppo degli standard comunicativi                           & 5                                                         \\ \cline{2-3} 
	\cellcolor{mainColorDark}{\color[HTML]{FFFFFF} }                                                                          & Realizzazione della barra di navigazione                       & 5                                                        \\ \cline{2-3} 
	\cellcolor{mainColorDark}{\color[HTML]{FFFFFF} }                                                                          & Realizzazione delle interfacce grafiche                        & 10                                                        \\ \cline{2-3} 
	\multirow{-4}{4cm}{\cellcolor{mainColorDark}{\color[HTML]{FFFFFF} \textbf{Definizione dell'interfaccia utente}}}            & \textbf{\textsc{Totale}}                                     & \textbf{20}                                               \\ \hline
	\cellcolor{mainColorDark}{\color[HTML]{FFFFFF} }                                                                          & Realizzazione delle pagine                                     & 30                                                        \\ \cline{2-3} 
	\cellcolor{mainColorDark}{\color[HTML]{FFFFFF} }                                                                          & Realizzazione delle interazioni tra le pagine                  & 12                                                        \\ \cline{2-3} 
	\cellcolor{mainColorDark}{\color[HTML]{FFFFFF} }                                                                          & Realizzazione e ottimizzazione dell'interazione                & 8                                                        \\ \cline{2-3} 
	\cellcolor{mainColorDark}{\color[HTML]{FFFFFF} }                                                                          & Realizzazione dei manuali                                      & 4                                                        \\ \cline{2-3} 
	\cellcolor{mainColorDark}{\color[HTML]{FFFFFF} }                                                                          & Produzione della versione $\alpha$                             & 2                                                        \\ \cline{2-3} 
	\multirow{-6}{4cm}{\cellcolor{mainColorDark}{\color[HTML]{FFFFFF} \textbf{Sviluppo}}}                                       & \textbf{\textsc{Totale}}                                     & \textbf{56}                                               \\ \hline
	\cellcolor{mainColorDark}{\color[HTML]{FFFFFF} }                                                                          & Alpha test e documento di test                                 & 10                                                        \\ \cline{2-3} 
	\cellcolor{mainColorDark}{\color[HTML]{FFFFFF} }                                                                          & Revisione del software                                         & 10                                                        \\ \cline{2-3} 
	\cellcolor{mainColorDark}{\color[HTML]{FFFFFF} }                                                                          & Beta test e documento di test                                  & 10                                                        \\ \cline{2-3} 
	\multirow{-4}{4cm}{\cellcolor{mainColorDark}{\color[HTML]{FFFFFF} \textbf{Test}}}                                           & \textbf{\textsc{Totale}}                                     & \textbf{30}                                               \\ \hline
	\cellcolor{mainColorDark}{\color[HTML]{FFFFFF} }                                                                          & Realizzazione copia master                                     & 2                                                        \\ \cline{2-3} 
	\cellcolor{mainColorDark}{\color[HTML]{FFFFFF} }                                                                          & Realizzazione delle copie per sviluppatori e commitente        & 2                                                        \\ \cline{2-3} 
	\multirow{-3}{4cm}{\cellcolor{mainColorDark}{\color[HTML]{FFFFFF} \textbf{Pubblicazione}}}                                  & \textbf{\textsc{Totale}}                                     & \textbf{4}                                               \\ \hline
\end{longtabu}

\section{Documento di pianificazione}
Il presente documento è stato modificato dopo circa una settimana di lavoro per poter includere le percentuali di completamento relative delle varie attività previste.

Le percentuali di completamento presenti in questa tabella (Tabella \ref{tab:docPianificazione}) sono percentuali empiriche basate su un calcolo approssimativo della mole di lavoro compiuta, che è poi stata paragonata alla mole di lavoro prevista per portare al termine una singola attività.
 
\begin{longtabu} to \textwidth {|X[4,L,m]|X[1,C,m]|X[1,C,m]|X[1,C,m]|}
	\caption{Costi in ore e percentuali di completamento delle attività previste durante la pianificazione}
	\label{tab:docPianificazione}\\
	\hline
	\rowcolor{mainColorDark}
	{\color[HTML]{FFFFFF} \textbf{\footnotesize \textsc{Attività}}} & {\color[HTML]{FFFFFF} \textbf{\footnotesize \textsc{Tempo stimato (ore)}}} & {\color[HTML]{FFFFFF} \textbf{\footnotesize \textsc{Tempo utilizzato (ore)}}} & {\color[HTML]{FFFFFF} \textbf{\footnotesize \textsc{Comple-\allowbreak{}tamento percentuale}}} \\ \hline
	\endhead
	\textbf{Acquisizione dei contenuti}               & x                                                            & x                                                               & x \%                                                                  \\ \hline
	\textbf{Verifica e validazione dei contenuti}     & x                                                            & x                                                               & x \%                                                                  \\ \hline
	\textbf{Definizione dell'interfaccia utente}      & x                                                            & x                                                               & x \%                                                                  \\ \hline
	\textbf{Raffinamento del materiale}               & x                                                            & x                                                               & x \%                                                                  \\ \hline
	\textbf{Sviluppo}                                 & x                                                            & x                                                               & x \%                                                                  \\ \hline
	\textbf{Test}                                     & x                                                            & x                                                               & x \%                                                                  \\ \hline
	\textbf{Pubblicazione}                            & x                                                            & x                                                               & x \%                                                                  \\ \hline
\end{longtabu}

\section{Risorse}
Di seguito, saranno illustrate tutte le risorse utilizzate per la realizzazione del sistema multimediale.
\subsection{Risorse umane}
La distribuzione del lavoro nel team di progettazione di sviluppo del sistema è stato diviso nel seguente modo:
\begin{itemize}
	\item Alessandro Annese: gestione e produzione degli elementi multimediali del sistema; supporto nella creazione delle pagine del sistema e nella gestione della documentazione.
	\item Davide De Salvo: gestione e produzione degli elementi multimediali del sistema; creazione delle pagine del sistema.
	\item Andrea Esposito: gestione della parte "\textit{backend}" dell'applicazione con una speciale attenzione all'utilizzo dei \textit{framework} necessari allo sviluppo dell'applicazione; gestione della documentazione e supporto nella creazione delle pagine del sistema.
	\item Graziano Montanaro: gestione e revisione dei contenuti testuali dell'applicazione; creazione delle pagine del sistema.
	\item Regina Zaccaria: gestione e revisione dei contenuti testuali dell'applicazione; creazione delle pagine del sistema.
\end{itemize}
Ovviamente, la suddivisione dei lavori precedentemente presentata non esclude la possibilità di variazioni successive o di collaborazioni fra membri del team con compiti differenti di collaborare nella risoluzione di un \textit{task} più complicato di quelli attualmente previsti.

\subsection{Risorse informative}
Tutte le informazioni riguardo gli strumenti musicali sono frutto di studi personali dei singoli componenti del team; mentre le informazioni relative alla musica saranno reperite da libri di testo o da esperti del settore.

\subsection{Risorse applicative}
Nello sviluppo dell'applicazione saranno utilizzati i seguenti applicativi:
\begin{itemize}
	\item Adobe Photoshop CC
	\item Adobe Illustrator CC
	\item Audacity
	\item 
\end{itemize}
Inoltre, si utilizzerà Git come sistema di controllo delle versioni, in combinazione con la piattaforma GitHub, che sarà usata per condividere i file sorgenti del sistema.

\subsection{Risorse strumentali}

\subsection{Risorse post-produzione}
Per la pubblicazione di \ProjectTitle{}, saranno necessari:
\begin{enumerate}
	\item Stesura dei contenuti
	\begin{itemize}
		\item Si prevede una spesa di circa 100€ (cento euro) per poter ricevere supporto nella stesura dei contenuti da parte di esperti del settore (maestri di musica dell'\emph{Accademia musicale \textbf{Francisco Tàrrega}}).
	\end{itemize}
\end{enumerate}

\section{Stima dei costi}
Non sono previste spese aggiuntive oltre a quelle per la stampa della documentazione cartacea.