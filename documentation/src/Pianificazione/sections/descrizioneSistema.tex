\section{Introduzione}
A seguito di un \textit{brainstorming} condotto dal gruppo \theteam{}, si è deciso di sviluppare una applicazione multimediale che ha l'obiettivo di avvicinare gli utenti alla musica e aiutarli nell'imparare a suonare uno strumento musicale di loro gradimento.

Tale idea è frutto di una difficile scelta fra una serie di idee concepite durante la sopraccitata fase di brainstorming. I criteri di scelta sono stati dettati dalle attitudini personali dei componenti del gruppo di lavoro, in composizione con le possibilità fornite dalla musica nel campo della multimedialità.

Il titolo ``\ProjectTitle{}'', scelto in quanto ritenuto accattivante, deriva dalla condensazione delle due parole \emph{``strumentale''}, che rimanda agli strumenti musicali, e \emph{``mente''}, che ricorda l'obiettivo finale del sistema: insegnare a suonare uno strumento.

L'applicazione sarà sviluppata in modo da esser rilasciata come sito web liberamente fruibile \emph{online}.

\section{Definizione dello scopo}
L'applicazione multimediale \ProjectTitle{} ha l'obiettivo di:
\begin{itemize}
	\item Avvicinare gli utenti alla musica
	\item Dare la possibilità agli utenti di imparare a suonare uno o più strumenti
	\item Dare la possibilità agli utenti di acquisire la capacità di riconoscere la natura degli accordi più comuni
	\item Fornire degli esercizi per valutare in autonomia l'apprendimento o per migliorare l'utilizzo dello strumento
\end{itemize}

\section{Il committente}
Il committente dell'applicazione è il docente del corso di Progettazione e Produzione Multimediale dell'anno 2018/19 dell'Università di Bari (CdL\footnote{CdL: abbreviazione di ``Corso di Laurea''.}: Informatica e Comunicazione Digitale), la Prof.ssa \emph{Rosa Lanzilotti}.

La consegna del sistema multimediale è stimata per il bimestre gennaio/febbraio 2019.

\section{Caratteristiche degli utenti}
Da una attenta fase di studio delle possibili tipologie di utente a cui l'applicazione è rivolta, si sono potute individuare tre classi fondamentali di utenti. Tali classi sono, tuttavia, flessibili: essendo basate sulla quantità di conoscenze nel campo musicale degli utenti, non è possibile determinare nette linee di separazione, bensì solo delle linee guida.

Le categorie di utenti individuate sono le seguenti:
\begin{itemize}
	\item \textbf{Neofita}: l'utente conosce nulla o quasi nulla della teoria musicale; non conosce alcuno strumento; potrà utilizzare l'applicazione per avvicinarsi al mondo della musica.
	\item \textbf{Intermedio}: l'utente conosce poco o nulla della teoria musicale; conosce, anche se poco, uno strumento musicale; potrà utilizzare l'applicazione per migliorare le proprie conoscenze sia in campo teorico che pratico.
	\item \textbf{Avanzato}: l'utente conosce almeno le basi della teoria musicale; conosce bene almeno uno strumento musicale; potrà utilizzare l'applicazione per migliorare le proprie conoscenze, conoscere nuovi strumenti e traslare le proprie conoscenze su altri strumenti.
\end{itemize}

Le classi di utenti sopra elencate sono descritte in tutte le loro caratteristiche nella seguente tabella (Tabella \ref{tab:utenti}). Si noti che le età che sono state associate a ogni categoria di utente sono frutto di una stima empirica e sono puramente indicative: le conoscenze in campo musicale non sono facilmente associabili a un'età media.
	\begin{longtabu}to \textwidth {|>{\columncolor{mainColorDark}}X[1,L,m]|X[1,L,m]|X[1,L,m]|X[1,L,m]|}
		\caption{Caratteristiche degli utenti.}
		\label{tab:utenti}\\
		\hline
		{\color[HTML]{FFFFFF} \textbf{}}                                    & \cellcolor{mainColorDark}{\color[HTML]{FFFFFF} \textbf{Neofita}}              & \cellcolor{mainColorDark}{\color[HTML]{FFFFFF} \textbf{Intermedio}}                & \cellcolor{mainColorDark}{\color[HTML]{FFFFFF} \textbf{Avanzato}}                                                          \\ \hline
		\endfirsthead
		\multicolumn{4}{c}{{\footnotesize\textbf{Tabella \thetable{}:} continuazione della pagina precedente.}}                                                                                                                                                                                                                                                                                      \\ \hline
		{\color[HTML]{FFFFFF} \textbf{}}                                    & \cellcolor{mainColorDark}{\color[HTML]{FFFFFF} \textbf{\textsc{Neofita}}}              & \cellcolor{mainColorDark}{\color[HTML]{FFFFFF} \textbf{\textsc{Intermedio}}}                & \cellcolor{mainColorDark}{\color[HTML]{FFFFFF} \textbf{\textsc{Avanzato}}}                                                          \\ \hline
		\endhead
		{\color[HTML]{FFFFFF} \textbf{Età}}                                 & Dai 13 anni in su                                                            & Dai 18 anni in su                                                                 & Dai 22 anni in su                                                                                                         \\ \hline
		{\color[HTML]{FFFFFF} \textbf{Livello di istruzione}}               & Intermedio                                                                   & Buono                                                                             & Ottimo                                                                                                                    \\ \hline
		{\color[HTML]{FFFFFF} \textbf{Capacità di lettura}}                 & Buona                                                                        & Buona                                                                             & Buona                                                                                                                     \\ \hline
		{\color[HTML]{FFFFFF} \textbf{Motivazione}}                         & Buona                                                                        & Buona                                                                             & Buona                                                                                                                     \\ \hline
		{\color[HTML]{FFFFFF} \textbf{Conoscenze preliminari}}              & Nessuna conoscenza teorico/pratica                                           & Scarse conoscenze teoriche/pratiche                                               & Buone conoscenze teoriche/pratiche                                                                                        \\ \hline
		{\color[HTML]{FFFFFF} \textbf{Abilità necessarie}}                  & Nessuna                                                                      & Nessuna                                                                           & Nessuna                                                                                                                   \\ \hline
		{\color[HTML]{FFFFFF} \textbf{Competenze informatiche}}             & Capacità basilari                                                            & Capacità basilari                                                                 & Capacità basilari                                                                                                         \\ \hline
		{\color[HTML]{FFFFFF} \textbf{Familiarità con il Web}}              & Capacità basilari                                                            & Capacità basilari                                                                 & Capacità basilari                                                                                                         \\ \hline
		{\color[HTML]{FFFFFF} \textbf{Capacità di digitazione e scrittura}} & Capacità basilari                                                            & Capacità basilari                                                                 & Capacità basilari                                                                                                         \\ \hline
		{\color[HTML]{FFFFFF} \textbf{Accesso a un computer}}               & Buona                                                                        & Buona                                                                             & Buona                                                                                                                     \\ \hline
		{\color[HTML]{FFFFFF} \textbf{Accesso a internet}}                  & Necessario                                                                   & Necessario                                                                        & Necessario                                                                                                                \\ \hline
		{\color[HTML]{FFFFFF} \textbf{Disponibilità (in tempo)}}            & Almeno tre ore consecutive al giorno                                         & Almeno due ore consecutive al giorno                                              & Disponibilità saltuaria                                                                                                   \\ \hline
		{\color[HTML]{FFFFFF} \textbf{Obiettivo dell'applicazione}}         & Insegnare le basi della musica, avvicinare l'utente all'uso di uno strumento & Migliorare le conoscenze dell'utente, fornirgli un supporto in caso di difficoltà & Assistere l'utente nell'ampliare le proprie conoscenze, fornirgli un supporto per trasporre le stesse su altri strumenti. \\ \hline
	\end{longtabu}

Come già anticipato precedentemente, il criterio con cui si sono individuate le seguenti classi di utenti si basa sulla classificazione dei livelli di conoscenza e competenza nel campo musicale degli stessi. Non è prevista alcuna classificazione basata sulle competenze di campo informatico, in quanto, come già specificato nella precedente tabella, non è richiesta alcuna abilità particolare in tale campo. Si noti, inoltre, che è necessario l'accesso a internet a prescindere dalla categoria d'utente: infatti, come già detto in precedenza, l'applicazione \ProjectTitle{} sarà sviluppata e sarà rilasciata come sito web e sarà quindi fruibile solo in presenza di una connessione alla rete.

\paragraph{Competenze informatiche necessarie} Come già anticipato, non sono necessarie delle competenze informatiche particolari per utilizzare l'applicazione: il sistema sarà progettato in modo da essere fruibile da qualsiasi utente che sappia interfacciarsi con un computer tramite \emph{browser} web. L'applicazione sarà quindi progettata per essere intuitiva e semplice da utilizzare. Saranno tuttavia fornite, eventualmente, delle scorciatoie da tastiera per i più esperti per velocizzare la loro navigazione all'interno dell'ipermedia.

\section{I vincoli}
In questa sezione del documento saranno presentati i vincoli da rispettare durante lo sviluppo dell'applicazione. Tali vincoli sono frutto di discussioni tra i membri del team e il commitente o di previsioni e obiettivi del team su delle fasi dello sviluppo successive.

\subsection{Requisiti minimi}
Il sistema multimediale sarà progettato per diverse piattaforme multimediali con i seguenti requisiti \emph{hardware} minimi:
\begin{itemize}
	\item Processore da 1228 Mhz
	\item RAM: 1 gb
	\item Hard Disk: da definire
	\item Touch screen presente
	\item Risoluzione dello schermo: $480 \times 800$
\end{itemize}
E con i seguenti requisiti minimi di software:
\begin{itemize}
	\item Browser web
\end{itemize}

\subsection{Budget}
L'applicazione multimediale è a scopo didattico quindi il committente non ha imposto nessun \emph{budget}.