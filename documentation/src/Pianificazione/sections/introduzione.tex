\section{Introduzione}
A seguito di un \textit{brainstorming} condotto dal gruppo \theteam{}, si è deciso di sviluppare una applicazione multimediale che ha l'obiettivo di avvicinare gli utenti alla musica e aiutarli nell'imparare a suonare uno strumento musicale di loro gradimento.

Tale idea è frutto di una difficile scelta fra una serie di idee concepite durante la sovracitata fase di brainstorming. I criteri di scelta sono stati dettati dalle attitudini personali dei componenti del gruppo di lavoro, in composizione con le possibilità fornite dalla musica nel campo della multimedialità

Il titolo "StrumentalMente", scelto in quanto ritenuto accattivante, deriva dalla condensazione delle due parole \emph{"strumentale"}, che rimanda agli strumenti musicali, e \emph{"mente"}, che ricorda l'obiettivo finale del sistema.

\section{Definizione dello scopo}
L’applicazione multimediale \ProjectTitle{} ha l’obiettivo di:
\begin{itemize}
	\item Avvicinare gli utenti alla musica
	\item Dare la possibilità agli utenti di imparare a suonare uno o più strumenti
	\item Acquisire la capacità di riconoscere la natura degli accordi più comuni
	\item Fornire degli esercizi per valutare in autonomia l’apprendimento o per migliorare l’utilizzo dello strumento
\end{itemize}

\section{Il committente}
Il committente dell’applicazione è il docente del corso di Progettazione e Produzione Multimediale dell’anno 2018/19 dell’Università di Bari (CdL\footnote{Corso di Laurea} Informatica e Comunicazione Digitale), la Dott.ssa \emph{Rosa Lanzilotti}.
\newline
La consegna del sistema multimediale è stata stimata per il bimestre gennaio/febbraio 2019.
